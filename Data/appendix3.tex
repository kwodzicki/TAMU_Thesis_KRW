% !TEX root = ../TAMU_Thesis_Main.tex
%%%%%%%%%%%%%%%%%%%%%%%%%%%%%%%%%%%%%%%%%%%%%%%%%%%
%
%  New template code for TAMU Theses and Dissertations starting Fall 2016.
%
%
%  Author: Sean Zachary Roberson 
%	 Version 3.16.09 
%  Last updated 9/12/2016
%
%  Modified 04 Nov. 2016 by Kyle R. Wodzicki
%%%%%%%%%%%%%%%%%%%%%%%%%%%%%%%%%%%%%%%%%%%%%%%%%%%

%%%%%%%%%%%%%%%%%%%%%%%%%%%%%%%%%%%%%%%%%%%%%%%%%%%%%%%%%%%%%%%%%%%%%%
%%                           APPENDIX B
%%%%%%%%%%%%%%%%%%%%%%%%%%%%%%%%%%%%%%%%%%%%%%%%%%%%%%%%%%%%%%%%%%%%%

\chapter{Bibliography Information}\label{appendix:02}

As previously mentioned, one program that can be used to organize references is \href{http://www.jabref.org}{\textbf{JabRef}}. While a tutorial of how to use \textbf{JabRef} is beyond the scope of this template, a brief discussion of how to use \textbf{BibTeX} follows.

\section{BibTeX}
After you have installed \textbf{JabRef}, or any citation manager of your choosing that is compatible with \textbf{BibTeX}, you must save a \textbf{BibTeX} database. This database file will contain all the information \textbf{BibTeX} requires to generate your bibliography. An example .bib file is include in this template that is named `myReference.bib'. The first entry of that file is shown below.

{\footnotesize \begin{verbatim}
@Article{Barn-JORVQ,
  author  = {Christopher F. Barnes and Richard L. Frost},
  title   = {Residual Vector Quantizers with Jointly Optimized Code Books},
  journal = {Advances in Electronics and Electron Physics},
  year    = {1992},
  volume  = {84},
  pages   = {1--59},
}
\end{verbatim}}

All of the entries in the entry are very self-explanatory, such as author and title, however, arguably the most important part of the entry is the key. The key is the first entry after @Article, which is Barn-JORVQ in this example. This is the key you will use in any cite commands for references, e.g.,

{\footnotesize\begin{verbatim}\cite{Barn-JORVQ}\end{verbatim}}.


Depending on the citation style that is used, there may be different cite commands for different types of in-text citations. It is important to know which commands must be used with the citation style you are using.

\section{Compiling with BibTeX}
When compiling your \LaTeX{} document, it is also important to remember to compile it twice to ensure that all equation, fig, table, etc. cross-references have updated correctly. However, when using citations from a .bib file in your document, the process is a little longer. To ensure that your bibliography generates correctly, one must run pdfLaTeX, then BibTeX, then pdfLaTeX twice. This will ensure that all the citations and cross-references are updated correctly. If you are using a program such as \textbf{MikTeX} or \textbf{ProTeXt}, this may be the default compilation method. However, if you \textbf{TeXShop} on a Mac, you must change the compiler, next to the Typeset button, from LaTeX to BibTeX, and back to compile properly. If compiling from command line, the sequence would be:

{\footnotesize \begin{verbatim}
pdflatex TAMU_Thesis_Main.tex
bibtex TAMU_Thesis_Main.aux
pdflatex TAMU_Thesis_Main.tex
pdflatex TAMU_Thesis_Main.tex
\end{verbatim}}

Be sure to check the output for any errors. If question marks (?) appear in any location where a reference should be, there was an issue with the compilation. Make certain that the key used in the cite command matches the corresponding references in the .bib file.