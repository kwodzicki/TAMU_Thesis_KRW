% !TEX root = ../TAMU_Thesis_Main.tex
%%%%%%%%%%%%%%%%%%%%%%%%%%%%%%%%%%%%%%%%%%%%%%%%%%%%%%%%%%%%%%%%%%%%%
%%                           ABSTRACT 
%
% Enter your abstract in the abstract environment.
%%%%%%%%%%%%%%%%%%%%%%%%%%%%%%%%%%%%%%%%%%%%%%%%%%%%%%%%%%%%%%%%%%%%%

\begin{abstract}

Our abstract goes here. Be sure to place the abstract in the abstract environment. And example of your abstract code may be:

\begin{verbatim}
\begin{abstract}
This is my abstract. It summaries my work.
\end{abstract}
\end{verbatim}

The abstract can either be in a separate \TeX{} file that is included using the \begin{verbatim}\include{}\end{verbatim} command, or it can be in the main document, but it MUST be in the abstract environment! This is true of all other sections before the table of contents, i.e., Dedication, Acknowledgements, and Nomenclature.
\end{abstract}
